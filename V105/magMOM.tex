\documentclass[titlepage=firstiscover, bibliography=totoc, captions=tableheading, parskip=half]{scrartcl}
\titlehead{
  \centering
  %\includegraphics[scale=2.3]{Bilder/logo.png}
}
\author{
  Alina Landmann, alina.landmann@tu-dortmund.de
  \and Jannine Salewski, jannine.salewski@tu-dortmund.de
}


\title{V501/V502 Ablenkung eines Elektronenstrahls im elektrischen und magnetischen Feld}
\date{Durchführung: 08.05.2018\\
Abgabe: 15.05.2018}
\publishers{TU Dortmund - Fakultät Physik}


\usepackage[aux]{rerunfilecheck}
\usepackage{polyglossia}
\setmainlanguage{german}
\usepackage{amsmath}
\usepackage{amssymb}
\usepackage{mathtools}
\usepackage{fontspec}
\usepackage[version=4]{mhchem}
\usepackage{scrhack}
\usepackage{float}
\floatplacement{table}{htbp}
\floatplacement{figure}{htbp}

\usepackage[locale=DE, separate-uncertainty=true, per-mode=symbol-or-fraction, decimalsymbol=.]{siunitx}
\usepackage{siunitx}
\sisetup{math-micro=\text{µ},text-micro=µ}
\newunit{\years}{a}
\newunit{\lightyears}{Lj}

\usepackage[style=alphabetic]{biblatex}
\addbibresource{lit.bib}

\usepackage[section, below]{placeins}
\usepackage[labelfont=bf,
font=small,
width=0.9\textwidth,
format=plain,
indention=1em]{caption}
\usepackage{graphicx}
\usepackage{grffile}
\usepackage{subcaption}

\usepackage[math-style=ISO, bold-style=ISO, sans-style=italic, nabla=upright, partial=upright]{unicode-math}
\setmathfont{Latin Modern Math}

\usepackage[autostyle]{csquotes}

\usepackage[unicode]{hyperref}

\usepackage{bookmark}

\usepackage{booktabs}

\usepackage{xcolor}

% Minipagepackage
\usepackage{minipage-marginpar}

\begin{document}

  \section{Zielsetzung}
  Messung des magnetischen Momentes durch drei verschiedene Methoden.

  \section{Theorie}
  Im Unterschied zum elektrischen Feld, besitzt das magnetische Feld keine Quellen. Es gibt nur magnetische Dipole, keine Monopole.
  Dipole erzeugen geschlossene Feldlinien. Dipole können makroskopisch durch einen Permanentmagneten oder durch eine
  stromdurchflossene Leiterschleife erzeugt werden. Die Leiterschleife besitzt dabei das magnetische Moment $\mu$,
  was mit folgender Formel berechnet werden kann:

  \begin{equation}
  \mu = I \cdot \vec{A}
  \end{equation}
  Hierbei ist $I$ der Strom, der durch die Leiterschleife fließt und $A$ die Querschnittsfläche der Leiterschleife.
  Im folgenden wird das magnetische Moment \mu eines Permanentmagneten, der sich im Innern einer Billardkugel befindet,
  experimentell mittels dreier verschiedener Methoden ermittelt.
  Zur Erzeugung eines homogenen Magnetfeldes wird im Experiment ein $Helmholtz-Spulenpaar$ verwendet. Hierbei sind
  zwei gleich große, gleich geformte, gleichsinnig vom Strom $I$ durchflossene Spulen so angeordnet, dass der Abstand
  zwischen den beiden Spulen annähernd ihrem Radius entspricht. Das somit vom Spulenpaar erzeugte Magnetfeld ist auf der
  Symmetrieachse der beiden Spulen homogen und kann mittels des $Biot-Savart-Gesetzes$ berechnet werden:

  \begin{equation}
  \vec{B}(x) = \frac{\mu_0 I}{4\pi} \frac{\symup{d}\vec{s}\times{\vec{r}}}{r^3}
  \end{equation}
  Ist der Abstand zwischen den beiden Spulen nicht identisch mit dem Radius der beiden Spulen, so gibt es zur Berechnung
  des Magnetfeldes folgende Formel, die für weitere Berechnungen verwendet wurde, da sich im Experiment der Radius der
  Spulen geringfügig von deren Abstand unterschied:

  \begin{equation}
  \vec{B}(x) = \frac{\mu_0 I}{2} \frac{R^2}{(R^2 + x^2)^\frac{3}{2}}
  \end{equation}
  Hierbei beschreibt $R$ den Spulenradius und der Abstand der beiden Spulen voneinander beträgt $2x$.
  Das gesamte Feld im Zentrum des Spulenpaares erfolgt durch Superposition der Einzelfelder.

  \begin{equation}
  B(0) = B_1(x) + B_1(-x) = \frac{\mu_0 I R^2}{(R^2 + x^2)^\frac{3}{2}}
  \end{equation}
  Hier wurde der Einfachheit halber das Zentrum des Spulenpaares in den Ursprung des Koordinatensystems gelegt.


  \section{Durchführung}
  Zur Vorbereitung auf den Versuch wurde die magnetische Flussdichte B im Zentrum des Helmholtz-Spulenpaares berechnet. Die gegebenen
  Werte zur Berechnung betrugen für den Radius der Spulen $R_\text{Spule}$ = $0.109 \, \mathrm{m}$, für den Abstand $d$ der beiden Spulen voneinader: $0.138 \mathrm{m}$
  für den Strom, der durch die Spulen floss: $I = 1 \, \mathrm{A}$ und die Windungszahl der beiden Spulen sollen $\symup{N} = 195$ sein.
  Zur Berechnung von B wurde Gleichung (4) verwendet.\\
  Das Ergebnis lautet: $B \approx \num{6.95e-6} \, T$.\\
  Außerdem wurde das Trägheitsmoment $J_\text{K}$ einer Kugel mit den Maßen $r_\text{K} = 2.5 \, \mathrm{cm}$, $m_K = 150 \, \mathrm{g}$ berechnet werden. Dieses beträgt $375 \, \si{\gram\per\centi\meter\squared}$.\\
  Es gilt nun, wie bereits erwähnt das magnetische Moment $\mu_\text{Dipol}$ einer Billardkugel, in der sich ein kleiner Permanentmagnet befindet auf drei verschiedene Arten
  zu ermitteln:
  \begin{enumerate}
  \item Unter Ausnutzung der Gravitation \\
  \item Unter Ausnutzung der Schwingungsdauer T \\
  \item Unter Ausnutzung der Präzessionsbewegung \\
  \end{enumerate}
  Zur Durchführung des Versuchs wurde ein Helmholtz-Spulenpaar mit oben angegeben Werten für $R_\text{Spule}$, $N$ und $d$ verwendet. Im Zentrum des Spulenpaares
  (beide kreisförmigen Spulen sind so angeordnet, dass sich das B-Feld senkrecht im Raum befindet) ist ein kleiner Messingzylinder befestigt, welcher
  eine kugelförmige Aussparung besitzt, sodass er die Billardkugel ideal aufnehmen kann. Mittels eines Luftkissens kann sich die Billardkugel
  reibungsfrei auf dem Zylinder bewegen. Das magnetische Moment der Billardkugel ist in Richtung eines kleinen Stiels gerichtet, welcher sich an der Kugel befindet.
  An der Oberseite des Spulenpaares befindet sich ein Stroboskop, welches im Verlauf des Versuchs zur Bestimmung der Frequenz der Drehbewegung genutzt werden wird.
  Der Strom und folglich auch das durch die Spulen erzeugte Magnetfeld, das Stroboskop und das Luftkissen können extern eingeschaltet werden.
  Veränderbar ist hier die Stromstärke, die Frequenz, mit der das Stroboskop aufblitzt und die Feldlinienrichtung, wobei diese im Experiment nicht
  verändert wurde. Sie war kontinuierlich auf "up" eingestellt, was besagt, dass die Feldlinien von unten nach oben ausgerichtet waren.\\
  Für alle weiteren Berechnungen wurden die Werte, als auch die Abmessungen für die Spulen und deren Abstand zueinander, die in der Versuchsvorbereitung
  gegeben waren, weiterverwendet. Sie wurden zu Beginn des Versuchs jedoch auf ihre Richtigkeit überprüft.
  \newpage\pagebreak\textbf{Bestimmung des magnetischen Moments der Billardkugel unter Ausnutzung der Schwerkraft} \\
  \\
  Bei dieser statischen Methode wird die Tatsache verwendet, dass eine Masse m, die der Schwerkraft $\vec{F} = \symup{m} \cdot \vec{g}$
  unterliegt, ein Drehmoment

  \begin{equation}
  \vec{D_g} = m \cdot (\vec{r} \times \vec{g})
  \end{equation}
  auf die Kugel ausübt. Das Gewicht, welches dieses Drehmoment ausüben soll,
  ist im Versuch verschiebbar auf einem Aluminiumstab, der als masselos anzusehen ist, befestigt. Dieser Stab wiederum steckt im Stiel der Billardkugel.
  Dem durch die Schwerkraft verursachten Drehmoment, wirkt das durch das B-Feld verursachte Drehmoment

  \begin{equation}
  D_B = \mu_\text{Dipol} \times \vec{B}
  \end{equation}
  entgegen.
  Bei genau einer Magnetfeldstärke sind $D_g$ und $D_B$ gleich und die Kugel führt keine Pendelbewegung mehr aus. Ist dies der Fall, so wird
  die eingestellte Stromstärke $I$ zur Berechnung des entstandenen B-Feldes, sowie der Abstand $r$, vom Mittelpunkt des Gewichts bis zur Billardkugel,
  notiert. Dieser Vorgang wird neun mal wiederholt, um statistische Fehler klein zu halten. Zum Schluss wird r gegen B aufgetragen und mittels einer linearen Regression
  das magnetische Moment $\mu_{\text{Dipol}}$ berechnet.\\
  \\

  \textbf{Bestimmung des magnetischen Momentes mit Hilfe der Schwingungsdauer T des Magneten} \\
  \\
  Versetzt man die Billardkugel im eingeschalteten B-Feld in Schwingung, so verhält sie sich wie ein harmonische Oszillator, dessen Bewegung wie folgt
  beschrieben werden kann:
  \begin{equation}
    -\vert{\vec{\mu}_\text{Dipol} \times \vec{\symup{B}}}\vert = J_\text{K} \cdot \frac{\symup{d^2}\theta}{\symup{d}t^2}
  \end{equation}
  Die Lösung der Differentialgleichung ergibt eine Gleichung für die Schwingungsdauer T:
  \begin{equation}
    \symup{T^2} = \frac{4\symup{\pi}^2 \symup{J}_\text{K}}{\mu_\text{Dipol}} \frac{1}{\symup{B}}
  \end{equation}
  Mithilfe der vorher berechneten Größen für das Trägheitsmoment der Kugel $symup{J}_\text{K}$ und das B-Feld lässt sich nun das magnetische Moment
  der Kugel berechnen.
  Im Experiment werden pro eingestellter Stromstärke zehn Periodendauern gemessen. Dies wird für neun verschiedene Stromstärken durchgeführt.
  Anschließend wird $\symup{T}^2$ gegen $\frac{1}{B}$ aufgetragen und mittels linearer Regression das magnetische Moment des Dipols $\mu_\text{Dipol}$ berechnet.\\
  \\

  \newpage\textbf{Bestimmung des magnetischen Moments über die Präzessionsbewegung der sich drehenden Billardkugel}\\
  \\
  Versetzt man die Billardkugel mit ihrem Stiel senkrecht nach oben zeigend in eine Rotationsbewegung und stößt sie danach mit einem kleinen Stoß gegen den Stiel
  an, sodass sie aus der senkrechten Position ausgelenkt wird, so führt die Achse der Kugel (durch den Stiel gehend) im eingeschalteten B-Feld eine Präzessionsbewegung aus.
  Dabei beschreibt die Achse der Kugel einen Kegelmantel um die Drehimpulsachse (senkrecht im Raum stehend). Durch die Rotation der Kugel bleibt deren Auslenkung stabil.
  Die Differentialgleichung für die Präzessionsbewegung sieht wie folgt aus:
  \begin{equation}
    \mu_\text{Dipol} \times \vec{\symup{B}} = \frac{\symup{d}\vec{L_\text{K}}}{\symup{d}t}
  \end{equation}
  Die Formel für die Präzessionsfrequenz $\symup{\Omega}_\text{p}$ ist eine Lösung der Differentielgleichung und lautet:
  \begin{equation}
    \Omega_\text{p} = \frac{\mu_\text{Dipol}\symup{B}}{\vert{\symup{L}_\text{K}\vert}}
  \end{equation}
  $\symup{L}_\text{K}$ beschreibt den Drehimpuls der ausgelenkten Kugel und kann mit Hilfe des Trägheitsmomentes der Kugel und deren Kreisfrequenz
  berechnet werden: $\symup{L}_\text{K} = \symup{J}_\text{K} \omega$ . \omega = 2 \pi \nu.
  Somit kann $\mu_\text{Dipol}$ über die Formel
  \begin{equation}
    \frac{1}{\symup{T}_\text{p}} = \frac{\mu_\text{Dipol}}{2\symup{\pi}\symup{L}_\text{K}} \symup{B}
  \end{equation}
  Um eine konstante Rotationsfrequenz $\nu$ zu erreichen, wird das Stroboskop eingeschaltet. Nachdem die Kugel bei senkrecht stehendem Stiel in eine
  Rotationsbewegung versetzt wurde, betrachte man den auf dem Stiel eingezeichneten kleinen Punkt in den regelmäßig aufleuchtenden Stroboskopblitzen.
  Erscheint der Punkt stationär, so hat die Billardkugel eine konstante Rotationsfrequenz und kann durch einen kleinen Stoß aus der senkrechten Achse
  ausgelenkt werden. Ist dies geschehen, so wird schnellstmöglich das B-Feld eingeschaltet und die Zeit $\symup{T}_\text{p}$ für einen Umlauf des Stiels
  gemessen werden. Um statistische Fehler zu minimieren, wird die Messung drei Mal pro Magnetfeldstärke und für insgesamt neun unterschiedliche
  Magnetfelder durchgeführt.
  Zur Auswertung wird $\frac{1}{\symup{T}_\text{p}}$ gegen die Magnetfeldstärke aufgetragen, um mittels linearer Regression $\mu_\text{Dipol}$ zu ermitteln.\\
  \\
  Zum Schluss werden die drei experimentell bestimmten Werte fuer $\mu_\text{Dipol}$ mit dem theoretischen Wert verglichen.








  \section{Auswertung}

  \section{Diskussion}

\end{document}
