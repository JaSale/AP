\section{Zielsetzung}
Im Versuch wird die Reichweite von Alphastrahlung in Luft ermittelt, sowie der statistische Zerfall analysiert.

\section{Theorie}
Die Entstehung von $\alpha$-Strahlung ist quantenmechanisch erklärbar. Die Kernkräfte und die Abstoßungskräfte der Protonen bilden einen unendlich
hohen Potentialwall. Klassisch ist eine Überwindung dessen nicht erklärbar. Quantenmechanisch besteht jedoch eine Tunnelwahrscheinlichkeit, die
das Überwinden des unendlich hohen Potentialwalls und somit die Entstehung von $\alpha$-Strahlung erklärt.
Beim Durchlaufen von Materie verliert $\alpha$-Strahlung ihre Energie. Dies ist durch drei verschiedene Prozesse zu erklären.
Zum einen durch die sogenannte Rutherford-Streuung, wobei es zu elastischen Stößen zwischen den $\alpha$-Teilchen und den Materieteilchen
kommt, welche für den Energieverlust und somit für den Versuch jedoch nur eine untergeordnete Rolle spielt.
Des weiteren ist ein Energieverlust durch Ionisationsprozesse und Anregung oder Dissoziation von Molekülen zu erklären.
Zu erwähnen ist, dass der Energieverlust der $\alpha$-Strahlung bei kleineren Geschwindigkeiten größer wird, was durch die längere
Zeit zu erklären ist, die die Teilchen zur Wechselwirkung miteinander haben. Außerdem ist der Energieverlust abhängig von der Energie der
$\alpha$-Teilchen selbst, sowie der Dichte des Materials mit dem die Strahlung wechselwirkt.
Die Bethe-Bloch-Gleichung beschreibt den Energieverlust von Teilchen, die eine hinreichend große Anfangsenergie besitzen. Sie verliert ihre
Gültigkeit für $\alpha$-Teilchen mit geringer Energie, weil dann Ladungsaustauschprozesse stattfinden.
\FloatBarrier
\begin{align*}
  - ~\frac{dE_\alpha}{dx} = \frac{z^2~e^4}{4\pi\Epsilon_0 m_e}~\frac{n ~ Z}{v^2}~ln\left(\frac{2 m_e v^2}{I}\right) .
\end{align*}
\FloatBarrier
In der Bethe-Bloch-Gleichung beschreibt $z$ die Ladung, $v$ die Geschwindigkeit der $\alpha$-Strahlung, $n$ die Teilchendichte, $Z$ steht für die
Ordnungszahl und $I$ für die Ionisationsenergie des Targetgases.
